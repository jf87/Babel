%!TEX root=../main.tex
\section{Related Work} % 2 columns
\label{sec:relwork}

% TODO delete this if not needed
% Categories:
%
% \begin{itemize}
%
%   \item Metadata generation
%
%   \begin{itemize}
%
%     \item What are other ways to generate metadata? (manual, machine learning, reg expressions...)
%
%   \end{itemize}
%
%   \item Metadata modeling, model generation, modeling??
%
%   \begin{itemize}
%
%     \item How are things modelled? Linked Data approach, common schema etc...
%
%   \end{itemize}
%
%   \item Building delivery/transfer process (Building commissioning)
%
%   \begin{itemize}
%
%     \item If a building is build, ther must be procedures in how it is beein transfered from the constructor to the architects, to the owner. These procedures will probably contain stuff like testing the light switches, BAS etc... It might be good to check it out and compare to our approach, take ideas from there etc...
%
%   \end{itemize}
%
% \end{itemize}

% Notes Whiteboard:
%
% Focus on the Building and Sensor Network Area
% - false positive + negative
% example: Arka, Dezhi, type recognizion (temp vs setpoint)
%
% Types of metadata:
%
% - structural
% - descriptive
%
%
% Different ways of meta data population
% automated vs manual
% --> manual
% - experts vs users
% How do we solve the problem of a strict format vs a user input based design?
% We cannot just use tags.
% Tagging is appropriate if the number of people is greater than the number of things to tag.
%
% We need to be 100\% conform, this means that the structural model needs to be given, some naming needs to be aggreed on.
% --> incentives for users
%
% Building commissioning:
%
% How does this fit into here?
% We do not the same, but...
% Maybe a way to bootstrap a ongoing commisioning.


\subsection{Metadata Modelling} % (fold)
\label{sub:metadata_modelling}

Metadata can be categorized in three categories: (i) administrative, (ii) structural and (iii) descriptive \cite{taylor2004organization}.

Implementations:

\begin{itemize}

  \item Semantic Web

  \begin{itemize}

    \item Linked Data
    \item RDF

  \end{itemize}
  
  \item Other ways?

\end{itemize}
% subsection metadata_modelling (end)

\subsection{Metadata Generation} % (fold)
\label{sub:metadata_generation}
% experts, authors, users, automated
Metadata population has been covered extensively in research for different forms of content (e.g., text, images, audio, videos).
Fields that have been dealing extensivly with the topic are: librarys and semantic web.

With YueLao, we are dealing with metdata in text form for buildings, a topic which has not gained much attention yet.
For text, automated metadata generation has been done in various ways, e.g. through natural language processing \cite{Yang2005}.
\cite{Yang2005} propose a machine learning approach to automatically generate metadata for the semantic web from the content of webpages.
The `content' for building set- and datapoints is the actuall reading of the points.
This reading is of some value in regards to metadata generation. We can derive the type of a point (e.g., temperature, light) from the range of its readings. 
In YueLao we use machine learning techniques to automate the discovering of set- and datapoint types.
(E.g., the user maps a thermostat setpoint and we check for correlation with other datapoints to conclude their type. The more devcices of the same type the user tags, the more certain becomes that.)
However, a mapping to an actual physical setpoint, to an actual location is not possible by looking at the `content'.
\cite{Rodriguez2008} use associate networks to transfer metadata from metadata-rich resources to metadata-poor ressources. They evaluate their system using a bibliographic dataset. 


Semi-automated/manual metadata generation is done in the so called folksonomies using e.g., community based tagging (\cite{Mathes2004} \cite{Golder2006}).
Tagging as user-generated metadata in web applications: \cite{budura2008tag}

% subsection metadata_generation (end)


\subsection{Building Comissioning} % (fold)
\label{sub:building_comissioning}
% Wikipedia: Building commissioning (Cx) is the process of verifying, in new construction, all (or some, depending on scope) of the subsystems for mechanical (HVAC), plumbing, electrical, fire/life safety, building envelopes, interior systems (example laboratory units), cogeneration, utility plants, sustainable systems, lighting, wastewater, controls, and building security to achieve the owner's project requirements as intended by the building owner and as designed by the building architects and engineers. Recommissioning is the methodical process of testing and adjusting the aforementioned systems in existing buildings.
Building comissioning is traditionaly done when a building is being transfered from the constructor/architect to the actual owner and operator. It assures that the building meets the requirements set by the building owner and is fully functioning. Functions that are commsioned are, e.g., HVAC, electrical and safety systems.
The ASHRAE Guideline 0-2013: The Commissioning Process defines it as follows \cite{ASHRAE:2013aa}:

``\emph{The Commissioning Process is a quality-focused process for enhancing the delivery of a project by verifying and documenting that the facility and all of its systems and assemblies are planned, designed, installed, tested, operated, and maintained to meet the Owner's Project Requirements.}''
% subsection building_comissioning (end)

If the building controls these functions through a BMS--like it is nearly always the case in recent buildings--, then working of the BMS is tested, 
In recent years continuous commisioning or recomissioning is becoming more common as a quality process, especially in the US (see e.g., XXX).
A recent study on the implementation of continuous commissioning has shown average savings of 0.29\$/sqft per year for schools, hospitals and office buildings in the US \cite{oh2014implemented}.
A meta-analysis published in \cite{mills2004cost}, has come to median commissioning costs of \$0.27/ft2, whole-building energy savings of 15 percent, and payback times of 0.7 years.