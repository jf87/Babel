%!TEX root=../main.tex
\section{Introduction} % 2 columns
Most commercial buildings in Europe and the US contain a Building Management System (BMS) \cite{Efficiency:2011aa}. \fixme{check source again}.
A BMS is a typically tightly coupled control and sensing system for a particular building. This system encompasses building functions like Heating, Ventilation, Air Conditioning (HVAC) and lighting \cite{salsbury2005survey}. \footnote{The two most common systems found today are LoneTalk and BACnet. Both have their roots in the 80s \cite{fisher2000bacnet}.}
BMS are increasingly seen and utilized in research and industry as an environment for smart applications.
Smart applications can allow users to interact directly with the BMS and its functions locally (e.g., a individual comfort application on the users smartphone) or enable building global, intelligent applications (e.g., match HVAC setting with room schedule system or implement demand response).

The applicability of these smart applications has already been showed extensively in past research:
\citeauthor{Erickson2012} developed a system that allowed users to directly communicate their thermal preferences to the BMS, leading to energy savings of 10\% and increased personal comfort \cite{Erickson2012}. In \cite{narayanaswamy2014data}, \citeauthor{narayanaswamy2014data} developed a system that can optimize energy usage and comfort by detecting anomalies in HVAC systems. \citeauthor{Fontugne2013} followed a similar approach in \cite{Fontugne2013}. They propose an unsupervised method to systematically detect abnormal energy consumption in buildings by finding inter-device relationships and reporting devices that deviate from the norm.

All that work has shown great potential in terms of improving both, (thermal) comfort and energy usage for existing buildings.
Experimental results for energy savings rank from XXX to XXX\%.
Applying these results to the overall commercial building sector in the US alone would result in yearly savings of XXX.
So why aren't these ideas used more broadly?

The problem is mainly a metadata problem. Metadata that describes the set- and data points of a BMS is used very inconsistently from building to building and even inside a distinctive building. Metadata is populated manually during the construction phase of a building and there is no common naming schema that is adhered to.
This has lead to a state where inconsistent, missing and wrong metadata is hindering an actual deployment of smart applications.
Moreover, non-residential buildings have typically life-cycles between 50 and 75 years \cite{khasreen2009life}.
It is therefore a inevitable that we will have to deal with their infrastructure for the foreseeable future.

The problem has been recognized in the BuildSys and broader computer science community.
\citeauthor{Dawson-haggerty2013} recognize the incredible importance of metadata for developing portable applications. They propose key-value metdata tags to describe datapoints, but don't further investigate a solution for the creation of these tags \cite{Dawson-haggerty2013}.
\citeauthor{Krioukov2012} propose a fuzzy query API and graph representation of building metadata. They manage to partly construct the metadata in some cases from the BMS.
Further, they propose computer vision techniques to extract VAC objects from a duct work diagram \cite{Krioukov2012}. \fixme{not sure how much of this should go into related work}

XXX proposes the idea of the semantic web to the building context, they however don't deal with the actual creation of that metadata. \fixme{approach from galway guys}
XXX proposes a system where metadata is semi-automatically completed using regular expressions to detect common patterns in metadata describers using the expertise input of the building manager. \fixme{arka's poster paper}
XXX proposes a system where ... \fixme{jorge dezhi}

Both approaches to metadata completion suffer an accuracy problem. They can help to make the BMS more understandable, but they don't achieve compatibility for applications between different buildings. This is why their work has been focused on sensor points, where a high accuracy and fuzzy-like matching might be sufficient.
Actuation points require however precise metadata to (i) overcome reservations by a building manager and (ii) enable a direct matching of actuation points to allow fine grained, correct control.

In this work we therefore present \emph{Yuelao: Simple Building Setpoint Matching and Metadata Labelling}.
The core contribution of our work is the design and implementation of a participatory, incremental system that allows for a gradual and distributed construction and matching of metadata.
We implement our system on two test sides in the US and in Denmark, using two distinct building management systems (BACnet and LoneTalk) to show the applicability, performance and accuracy of our system.
We further develop two sample smart applications, a personal comfort control app and a ???,  that make use of this incrementally constructed model and that are transferable between buildings without any modification to the application code.

Our secondary contribution is the establishment of an extensible component library that we hope will be used and extended by others.
All our work is open sourced under a BSD license and can be found on ???.

The remainder of this paper is structured as follows: First we presents the state of the art and related work, including an analysis of the applicability of these technologies in the context of commercial buildings.
This is the basis for our succeeding description of system design and implementation.
We then presents the deployment and evaluation of our system, concluded with an outlook on future work.

% Building Management Systems (BMS) are increasingly seen as a target for smart ap
% Over the past years we have seen an tremendous growth in an area which is often marketed as the Internet of Things (IoT).
% A great amount of these technologies is targeting buildings with the promise of energy savings and comfort improvements.
% Technologies with such promises are e.g., smart lighting and smart thermostats.
% Non residential buildings however contain usually a Building Management System (BMS). A BMS is a typically tightly coupled control and sensing system for a particular building. This system encompasses building functions like Heating, Ventilation, Air Conditioning (HVAC) and lighting \cite{salsbury2005survey}. \footnote{The two most common systems today are LONEWORKS and BACnet. Both have their roots in the 80s \cite{fisher2000bacnet}.}
%
% The tight coupling leads to silos of information and control. A building control running on one building cannot just be transferred to to another building. There is no consistent naming of setpoints and sensor sources between buildings, yet for setpoints and sensor sources inside a single building.
% Non-residential buildings have typically life-cycles between 50 and 75 years \cite{khasreen2009life}.
% It is therefore a inevitable that we will have to deal with their infrastructure for the foreseeable future.
% This environment makes it very hard to enable smart applications, smart applications that might make use of both, a traditional BAS and new smart devices that arrive with IoT.
%
% Intelligence in buildings is largely driven by interoperability between physical and virtual setpoints. Metadata describes the attributes of the setpoints, and is critical for applications to interact with them.
% What is needed is a meta data schema that is consistent inside one building, but also across different buildings.
% Only consistent meta data enables even the most simple applications.
%
% In this work we therefore present \emph{Babel: Simple Building Setpoint Matching and Metadata Labelling}.
% The core contribution of our work is the design and implementation of a participatory system that allows for a gradual and distributed construction and matching of metadata.
% We implement our system on two test sides in the US and in Denmark, using two distinct building management systems (BACnet and Loneworks) to show the applicability, performance and accuracy of our system.
% We further develop two sample smart applications, a personal comfort control app and a ???,  that make use of this model and that are transferable between buildings without any modification to their application code.
% Our secondary contribution is the establishment of an extensible component library that we hope will be used and extended by others.
% All our work is open sourced under a BSD license and can be found on ???.
%
% The remainder of this paper is structured as follows: First we presents the state of the art and related work, including an analysis of the applicability of these technologies in the context of non-residential buildings.
% This is the basis for our succeeding description of system design and implementation.
% We then presents the deployment and evaluation of our system, concluded with an outlook on future work.
%
% \cite{bushby1997bacnet}