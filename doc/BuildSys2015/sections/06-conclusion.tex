%!TEX root=../main.tex
\section{Conclusion}  % 0.5 columns
\label{sec:conclusion}

This is awesome!


\subsection{Future Work} % (fold)
\label{sub:future_work}

\subsubsection{User Study} % (fold)
\label{ssub:user_study}
Do some study where we actually give this application to users.
% subsubsection user_study (end)

\subsubsection{Implement as a Game} % (fold)
\label{ssub:implement_as_a_game}
Make it fun to use.
% subsubsection implement_as_a_game (end)

\subsubsection{Idea: Participatory, CS based building recommissioning} % (fold)
\label{ssub:idea_participatory_cs_based_building_recommissioning}

Problem:
Buildings are often commissioned only once. Their original functioning diverges over time from the optimum.
This results in basically all buildings not used to their potential. Energy is wasted, people are uncomfortable.

Re-Commissioning and continuous commissioning can fix that and achieve big improvements.
However, this re-commissioning process is mainly done manually and therefore costly.
Also it needs to be done continuously if a building should not fall back into a sub-optimal state.

We maybe can extend the idea of participatory linking of physical and digital points by some form of participatory, automated commissioning.
After having matched points with each other,  we can use this model to keep the building in a good state.
E.g., we have the Android app, people can tell: too hot, too cold which we can use as the input for the HVAC system for optimizing on comfort.
Or we can use it to tell that something is broken.

% subsubsection idea_participatory_cs_based_building_recommissioning (end)

% subsection future_work (end)